\documentclass[a4paper,12pt]{article}

\usepackage[utf8]{inputenc}
\usepackage{pmboxdraw}
\usepackage[spanish]{babel}
\usepackage{graphicx}
\graphicspath{{../img/}}
\usepackage[section]{placeins}
\usepackage{hyperref}
\hypersetup{
    colorlinks = true,
    linkcolor = blue,
    urlcolor = cyan
}

\title{\textbf{Afondamento nas Competencias Profesionais}\\[1em]\textbf{Actividad 1.3 - Arquitectura}\\[1em]Documentación de código: Express API}
\author{Daniel Calvar Cruz, 2ºCS DAM}
\date{}

\begin{document}

\maketitle
\clearpage
\hypertarget{anchor-indice}{}
\tableofcontents
\newpage

% ==============================
\section{Descripción general}
\hyperlink{anchor-indice}{\textbf{Volver}}\\

Es una API basada en el modelo estándar de aplicaciones web, derivado del clásico Vista-Controlador.

Al ser una API pequeña tiene un funcionamiento muy simple, se ocupa de llevar a cabo acciones CRUD con una base de datos MongoDB y devolver al frontend los datos pertinentes.

\clearpage

% ==============================
\section{Información técnica y dependencias}
\hyperlink{anchor-indice}{\textbf{Volver}}\\



\begin{itemize}
    \item \textbf{Versión Node.js:} 22.14
    \item \textbf{Base de datos:} MongoDB Atlas
    \item \textbf{Dependencias NPM:}
    \begin{itemize}
        \item \textbf{cors:} 2.8.5
        \item \textbf{dotenv:} 16.4.7
        \item \textbf{express:} 4.21.1
        \item \textbf{mongodb:} 6.12.0
        \item \textbf{mongoose:} 8.13.2
        \item \textbf{nodemon:} 3.1.10  
    \end{itemize}
\end{itemize}
\clearpage

% ==============================
\section{Configuración}
\hyperlink{anchor-indice}{\textbf{Volver}}\\

\subsection{Variables de entorno (.env)}
\begin{itemize}
    \item \texttt{MONGO\_URI}: URI de conexión a MongoDB Atlas o local.
    \item \texttt{PORT}: Puerto en el que se ejecutará el servidor.
    \item \texttt{DDBB\_NAME}: Nombre de la base de datos en Mongo Atlas.
\end{itemize}

\subsection{Archivo vercel.json}
Define la configuración para el despliegue en Vercel, incluyendo las rutas y la carpeta de salida de la API.

\subsection{CORS}
El middleware \texttt{cors.js} define los dominios permitidos para acceder a la API, garantizando la seguridad y evitando errores de origen cruzado.

\clearpage

% ==============================
\section{Estructura}
\hyperlink{anchor-indice}{\textbf{Volver}}\\

\begin{verbatim}
├── api/
│   ├── config/
│   │   └── ddbb.js
│   ├── controllers/
│   │   └── crudController.js
│   ├── middlewares/
│   │   └── cors.js
│   ├── models/
│   │   ├── Module.js
│   │   └── index.js
│   ├── routes/
│   │   ├── crud.js
│   │   └── index.js
│   ├── services/
│   │   └── crudServices.js
│   ├── app.js
│   └── server.js
├── node_modules
├── package-lock.json
├── package.json
└── vercel.json
\end{verbatim}

\begin{itemize}
    \item (root) node-modules: paquetes de node.
    \item (root) package-lock.json / package.json: archivos de configuración.
    \item (root) vercel.json: configuración de despliegue.
    \item (api) config: conexión a la base de datos.
    \item (api) middlewares: middlewares del sistema.
    \item (api) models: esquemas de Mongoose.
    \item (api) routes: rutas de entrada de los endpoints.
    \item (api) controllers: enlaza las rutas de entrada con los servicios.
    \item (api) services: lógica de negocio.
    \item (api root) app: configuración del servidor Express.
    \item (api root) server: punto de entrada y conexión del servidor.
\end{itemize}

\clearpage

% ==============================
\section{Modelo de datos}
\hyperlink{anchor-indice}{\textbf{Volver}}\\

El modelo principal se define en \texttt{models/Module.js} y representa una ficha de personaje.\\

\begin{tabular}{|c|c|c|}
    \hline
    \textbf{Campo} & \textbf{Tipo} & \textbf{Descripción} \\
    \hline
    nombre & String & Nombre del personaje \\
    \hline
    clase & String & Clase o rol del personaje \\
    \hline
    HP & Number & Puntos de vida \\
    \hline
    indice & Number & Índice interno autogenerado \\
    \hline
\end{tabular}\\[1em]

Ejemplo de documento JSON:
\begin{verbatim}
{
  "nombre": "Mustakrakish",
  "clase": "Trol del lago",
  "HP": 9999,
  "indice": 1
}
\end{verbatim}

\clearpage

% ==============================
\section{Arquitectura}
\hyperlink{anchor-indice}{\textbf{Volver}}\\

La API sigue una arquitectura modular basada en el patrón MVC (Modelo–Vista–Controlador).

\begin{itemize}
    \item \textbf{Rutas:} reciben la petición HTTP y determinan qué controlador usar.
    \item \textbf{Controladores:} validan y gestionan las peticiones, llamando a los servicios necesarios.
    \item \textbf{Servicios:} contienen la lógica de negocio y se comunican con la base de datos.
    \item \textbf{Modelos:} definen la estructura de los datos en MongoDB mediante Mongoose.
\end{itemize}

Flujo de ejecución de una petición:
\begin{verbatim}
Cliente → Ruta → Controlador → Servicio → Modelo (MongoDB) → Respuesta
\end{verbatim}

\clearpage

% ==============================
\section{API}
\hyperlink{anchor-indice}{\textbf{Volver}}\\

\subsection{Endpoints detallados}

% -------------------------------------------
\subsubsection{/create}
Crea una nueva ficha en la base de datos. Crea el índice automáticamente.\\

Tipo: \textbf{POST}\\[0.5em]

\textbf{Body:}\\

\begin{tabular}{|c|c|c|}
    \hline
    \textbf{Parámetro} & \textbf{Tipo}  & \textbf{Descripción}\\ 
    \hline
    nombre & String & Nombre del personaje de la ficha\\ 
    \hline
    clase & String & Clase del personaje de la ficha\\ 
    \hline
    HP & int & Puntos de vida del personaje\\ 
    \hline
\end{tabular}\\[1em]

\textbf{Respuesta:}\\

\begin{tabular}{|c|c|c|}
    \hline
    \textbf{Parámetro} & \textbf{Tipo}  & \textbf{Descripción}\\ 
    \hline
    type & String & success / failure\\ 
    \hline
    message & String & Mensaje de éxito o fallo\\ 
    \hline
\end{tabular}\\[1em]

\textbf{Ejemplo completo de uso:}
\begin{verbatim}
POST /create
{
  "nombre": "Ragnar",
  "clase": "Guerrero",
  "HP": 100
}

Respuesta:
{
  "type": "success",
  "message": "Ficha creada correctamente"
}
\end{verbatim}

\FloatBarrier
% -------------------------------------------
\subsubsection{/read}
Devuelve un documento en concreto.\\

Tipo: \textbf{POST}\\[0.5em]

\textbf{Body:}\\

\begin{tabular}{|c|c|c|}
    \hline
    \textbf{Parámetro} & \textbf{Tipo}  & \textbf{Descripción}\\ 
    \hline
    filtroKey & String & Clave a buscar en base de datos\\ 
    \hline
    filtroValue & String & Valor que va a buscar usando la clave\\ 
    \hline
\end{tabular}\\[1em]

\textbf{Respuesta:}\\

\begin{tabular}{|c|c|c|}
    \hline
    \textbf{Parámetro} & \textbf{Tipo}  & \textbf{Descripción}\\ 
    \hline
    type & String & success / failure\\ 
    \hline
    message & String & Mensaje de éxito o fallo\\ 
    \hline
    data & Objeto JSON & Datos del documento encontrado\\ 
    \hline
\end{tabular}\\[1em]

\textbf{Ejemplo completo de uso:}
\begin{verbatim}
POST /read
{
  "filtroKey": "nombre",
  "filtroValue": "Ragnar"
}

Respuesta:
{
  "type": "success",
  "message": "Documento encontrado",
  "data": {
    "nombre": "Ragnar",
    "clase": "Guerrero",
    "HP": 100,
    "indice": 1
  }
}
\end{verbatim}

\FloatBarrier
% -------------------------------------------
\subsubsection{/update}
Modifica un documento en concreto.\\

Tipo: \textbf{POST}\\[0.5em]

\textbf{Body:}\\

\begin{tabular}{|c|c|c|}
    \hline
    \textbf{Parámetro} & \textbf{Tipo}  & \textbf{Descripción}\\ 
    \hline
    indice & int & Índice interno en base de datos de la ficha\\ 
    \hline
    nombre & String & Nombre del personaje\\ 
    \hline
    clase & String & Clase del personaje\\ 
    \hline
    HP & int & Puntos de vida del personaje\\ 
    \hline
\end{tabular}\\[1em]

\textbf{Respuesta:}\\

\begin{tabular}{|c|c|c|}
    \hline
    \textbf{Parámetro} & \textbf{Tipo}  & \textbf{Descripción}\\ 
    \hline
    type & String & success / failure\\ 
    \hline
    message & String & Mensaje de éxito o fallo\\ 
    \hline
\end{tabular}\\[1em]

\textbf{Ejemplo completo de uso:}
\begin{verbatim}
POST /update
{
  "indice": 1,
  "nombre": "Ragnar",
  "clase": "Berserker",
  "HP": 130
}

Respuesta:
{
  "type": "success",
  "message": "Ficha actualizada correctamente"
}
\end{verbatim}

\FloatBarrier
% -------------------------------------------
\subsubsection{/delete}
Borra un documento en concreto.\\

Tipo: \textbf{POST}\\[0.5em]

\textbf{Body:}\\

\begin{tabular}{|c|c|c|}
    \hline
    \textbf{Parámetro} & \textbf{Tipo}  & \textbf{Descripción}\\ 
    \hline
    indice & int & Índice interno en base de datos de la ficha\\ 
    \hline
\end{tabular}\\[1em]

\textbf{Respuesta:}\\

\begin{tabular}{|c|c|c|}
    \hline
    \textbf{Parámetro} & \textbf{Tipo}  & \textbf{Descripción}\\ 
    \hline
    type & String & success / failure\\ 
    \hline
    message & String & Mensaje de éxito o fallo\\ 
    \hline
\end{tabular}\\[1em]

\textbf{Ejemplo completo de uso:}
\begin{verbatim}
POST /delete
{
  "indice": 1
}

Respuesta:
{
  "type": "success",
  "message": "Ficha eliminada correctamente"
}
\end{verbatim}

\FloatBarrier
% -------------------------------------------
\subsubsection{/readall}
Devuelve todos los documentos de la base de datos.\\

Tipo: \textbf{GET}\\[0.5em]

\textbf{Respuesta:}\\

\begin{tabular}{|c|c|c|}
    \hline
    \textbf{Parámetro} & \textbf{Tipo}  & \textbf{Descripción}\\ 
    \hline
    type & String & success / failure\\ 
    \hline
    message & String & Mensaje de éxito o fallo\\ 
    \hline
    data & Array de objetos JSON & Lista de todos los documentos encontrados\\ 
    \hline
\end{tabular}\\[1em]

\textbf{Ejemplo completo de uso:}
\begin{verbatim}
GET /readall

Respuesta:
{
  "type": "success",
  "message": "Documentos encontrados",
  "data": [
    {
      "nombre": "Ragnar",
      "clase": "Berserker",
      "HP": 130,
      "indice": 1
    },
    {
      "nombre": "Luna",
      "clase": "Maga",
      "HP": 80,
      "indice": 2
    }
  ]
}
\end{verbatim}

\FloatBarrier


\clearpage

% ==============================
\section{Pruebas}
\hyperlink{anchor-indice}{\textbf{Volver}}\\

\begin{itemize}
    \item Para probar la API localmente:
    \begin{verbatim}
    npm run dev
    \end{verbatim}
    \item Recomendado usar herramientas como \textbf{Postman} o la extensión de VS Code \textbf{Thunderclient}.
    \item Ejemplo:
    \begin{verbatim}
    POST http://localhost:3000/create \
    Body '{"nombre":"Luna","clase":"Maga","HP":80}'
    \end{verbatim}
    \item Respuesta esperada:
    \begin{verbatim}
    {
      "type": "success",
      "message": "Ficha creada correctamente"
    }
    \end{verbatim}
\end{itemize}

\clearpage

% ==============================
\section{Despliegue}
\hyperlink{anchor-indice}{\textbf{Volver}}\\

\begin{itemize}
    \item La API se despliega en \textbf{Vercel}.
    \item Se usa el archivo \texttt{vercel.json} para definir las rutas y la carpeta raíz.
    \item Variables de entorno configuradas desde el panel de Vercel.
    \item Ejemplo de endpoint:
    \begin{verbatim}
    https://miapi.vercel.app/api/readall
    \end{verbatim}
\end{itemize}

\clearpage

% ==============================
\section{Futuras mejoras}
\hyperlink{anchor-indice}{\textbf{Volver}}\\

\begin{itemize}
    \item Implementar autenticación JWT para proteger endpoints.
    \item Añadir validación de datos con \texttt{express-validator}.
    \item Centralizar manejo de errores.
    \item Incluir tests automatizados (Jest o Mocha).
    \item Generar documentación interactiva con Swagger.
\end{itemize}

\clearpage

\end{document}
