\documentclass[a4paper,12pt]{article}

\usepackage[utf8]{inputenc}
\usepackage{pmboxdraw}
\usepackage[spanish]{babel}
\usepackage{graphicx}
\graphicspath{{../img/}}
\usepackage[section]{placeins}
\usepackage{hyperref}
\hypersetup{
    colorlinks = true,
    linkcolor = blue,
    urlcolor = cyan
}

\title{\textbf{Afondamento nas Competencias Profesionais}\\[1em]\textbf{Actividad 1.3 - Arquitectura}\\[1em]Análisis de código: Express API}
\author{Daniel Calvar Cruz, 2ºCS DAM}
\date{}

\begin{document}

\maketitle
\clearpage
\hypertarget{anchor-indice}{}
\tableofcontents
\newpage

\section{Descripción general}
\hyperlink{anchor-indice}{\textbf{Volver}}\\

Es una API basada en el modelo estándar de aplicaciones web, derivado del clásico Vista-Controlador.

Al ser una API pequeña tiene un funcionamiento muy simple, se odupa de llevar a cabo acciones CRUD con una base de datos Mongodb y devolver al frontend los datos pertinentes.


\section{Estructura}
\hyperlink{anchor-indice}{\textbf{Volver}}\\

\begin{verbatim}
    ├── api/
│   ├── config/
│   │   └── ddbb.js
│   ├── controllers/
│   │   └── crudController.js
│   ├── middlewares/
│   │   └── cors.js
│   ├── models/
│   │   ├── Module.js
│   │   └── index.js
│   ├── routes/
│   │   ├── crud.js
│   │   └── index.js
│   ├── services/
│   │   └── crudServices.js
│   ├── app.js
│   └── server.js
├── node_modules
├── package-lock.json
├── package.json
└── vercel.json
\end{verbatim}

\begin{itemize}
    \item (root) node-modules: paquetes de node.
    \item (root) package-lock.json: archivo de configuración.
    \item (root) package.json: archivo de configuración.
    \item (root) vercel.json: archivo de configuración vercel.
    \item (api) config: conexión a base de datos.
    \item (api) middlewares: middlewares del sistema.
    \item (api) models: interfaces de Mongoose.
    \item (api) routes: rutas de entrada de los endpoints.
    \item (api) controllers: enlaza las rutas de entrada con los servicios.
    \item (api) services: lógica de negocio.
    \item (api root) app: configuración del servidor Express.
    \item (api root) server: punto de entrada y conexión del servidor.
\end{itemize}


\clearpage

\section{API}
\hyperlink{anchor-indice}{\textbf{Volver}}\\

\subsection{Endpoints}

\subsubsection{/create}
Crea una nueva ficha en la base de datos. Crea el índice automáticamente.\\

Tipo: POST\\

Body:\\

\begin{tabular}{|c|c|c|}
    \hline
    \textbf{Parámetro} & \textbf{Tipo}  & \textbf{Descripción}\\ 
    \hline
    nombre & String & Nombre del personaje de la ficha\\ 
    \hline
    clase & String & Nombre la clase del personaje de la ficha\\ 
    \hline
    HP & int & Hit points del personaje de la ficha\\ 
    \hline
\end{tabular}\\

Respuesta:\\

\begin{tabular}{|c|c|c|}
    \hline
    \textbf{Parámetro} & \textbf{Tipo}  & \textbf{Descripción}\\ 
    \hline
    type & String & success / failure\\ 
    \hline
    message & String & Mensaje de éxito o fallo\\ 
    \hline
\end{tabular}

\subsubsection{/read}
Devuelve un documento en concreto.\\

Tipo: POST\\

Body:\\

\begin{tabular}{|c|c|c|}
    \hline
    \textbf{Parámetro} & \textbf{Tipo}  & \textbf{Descripción}\\ 
    \hline
    filtroKey & String & Clave a buscar en base de datos\\ 
    \hline
    filtroValue & String & Valor que va a buscar usando la clave para recuperar el documento\\ 
    \hline
\end{tabular}\\

Respuesta:\\

\begin{tabular}{|c|c|c|}
    \hline
    \textbf{Parámetro} & \textbf{Tipo}  & \textbf{Descripción}\\ 
    \hline
    type & String & success / failure\\ 
    \hline
    message & String & Mensaje de éxito o fallo\\ 
    \hline
    data & Objeto JSON & Datos del documento encontrado (si lo encuentra)\\ 
    \hline
\end{tabular}\\

\subsubsection{/update}
Modifica un documento en concreto.\\

Tipo: POST\\

Body:\\

\begin{tabular}{|c|c|c|}
    \hline
    \textbf{Parámetro} & \textbf{Tipo}  & \textbf{Descripción}\\ 
    \hline
    indice & int & Indice interno en base de datos de la ficha\\ 
    \hline
    nombre & String & Nombre del personaje de la ficha\\ 
    \hline
    clase & String & Nombre la clase del personaje de la ficha\\ 
    \hline
    HP & int & Hit points del personaje de la ficha\\ 
    \hline
\end{tabular}\\

Respuesta:\\

\begin{tabular}{|c|c|c|}
    \hline
    \textbf{Parámetro} & \textbf{Tipo}  & \textbf{Descripción}\\ 
    \hline
    type & String & success / failure\\ 
    \hline
    message & String & Mensaje de éxito o fallo\\ 
    \hline
\end{tabular}\\

\subsubsection{/delete}
Borra un documento en concreto.\\

Tipo: POST\\

Body:\\

\begin{tabular}{|c|c|c|}
    \hline
    \textbf{Parámetro} & \textbf{Tipo}  & \textbf{Descripción}\\ 
    \hline
    indice & int & Indice interno en base de datos de la ficha\\ 
    \hline
\end{tabular}\\

Respuesta:\\

\begin{tabular}{|c|c|c|}
    \hline
    \textbf{Parámetro} & \textbf{Tipo}  & \textbf{Descripción}\\ 
    \hline
    type & String & success / failure\\ 
    \hline
    message & String & Mensaje de éxito o fallo\\ 
    \hline
\end{tabular}\\

\subsubsection{/readall}
Devuelve todos los documentos de la base de datos.\\

Tipo: GET\\

Respuesta:\\

\begin{tabular}{|c|c|c|}
    \hline
    \textbf{Parámetro} & \textbf{Tipo}  & \textbf{Descripción}\\ 
    \hline
    type & String & success / failure\\ 
    \hline
    message & String & Mensaje de éxito o fallo\\ 
    \hline
    data & Array de objetos JSON & Lista de todos los documentos de la base de datos (si los encuentra)\\ 
    \hline
\end{tabular}\\

\subsection{nueva sub section APIS}

\end{document}
